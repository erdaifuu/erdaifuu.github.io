\section{Lectures 21-24}
\subsection{Row Space}
\begin{definition}
  The \vocab{row space} of an $m \times n$ matrix $A$ (with real entries) is the span of the rows of $A$
  each viewed as a vector in $\RR^{n}$.
\end{definition}

\begin{definition}
  The \vocab{row rank} of a matrix is the dimension of its row space.
\end{definition}

\begin{theorem}
  If $B$ is a matrix in row-reduced form, then a basis for the row space of $B$ is the set of nonzero rows 
  of $B$, and the row rank is the number of nonzero rows of $B$.
\end{theorem}

\begin{theorem}
  If matrix $B$ is obtained from matrix $A$ by a series of elementary row operations, then 
  \begin{equation*}
    rowspace(B) = rowspace(A).
  \end{equation*}
\end{theorem}

\begin{lemma}
  If $V$ is a vector space and $S \subset T \subset V$, then $span(S) \subset span(V)$.
\end{lemma}

\begin{lemma}
  If $V$ is a vector space and $S \subset V$, then $span(span(S)) = span(S)$.
\end{lemma}

\begin{lemma}
  If $V$ is a vector space and $S, T \subset V$ such that $S \subset span(T0)$, then $span(S) \subset span(T)$.
\end{lemma}

\begin{lemma}
  If $v_{1}, \dots, v_{n}$ are nonzero vectors in the vector space $V$, then
  \begin{enumerate}
    \item $span\{v_{1}, \dots, v_{k}, \dots, v_{n}\} = span \left\{v_{1}, \dots, v_{k - 1}, v_{k} + \lambda v, \dots, v_{n}\right\}$.
    \item $span \left\{v_{1}, \dots, v_{k}, \dots, vln\right\} = span \left\{v_{1}, \dots, v_{k - 1}, \lambda v_{k}, \dots, v_{n}\right\}$.
  \end{enumerate}
\end{lemma}

\begin{theorem}
  If $B$ is a matrix in row-reduced form, then a basis for the row space of $B$ is the set of nonzero rows of $B$, and 
  the row rank is the number of nonzero rows of $B$.
\end{theorem}

\begin{theorem}
  If matrix $B$ is obtained from matrix $A$ by a series of elementary row operations, then $rowspace(B) = rowspace(A)$.
\end{theorem}

\begin{corollary}
  A basis for the row space of a matrix $A$ is the set of nonzero rows of a matrix $B$ obtained by transforming $A$ to row-reduced form using
  elementary row operations; the row rank of $A$ is the number of nonzero rows of the row-reduced matrix $B$.
\end{corollary}

Keep in mind that a basis for $span(S)$ equals a basis for $rowspace(A)$ equals a basis for $rowspace(B)$ equals the set of nonzero rows of $B$,
where $S$ is the row space, $A$ is the original matrix, and $B$ is the row reduced form of $A$.

To find a basis for the span of $\left\{u_{1}, \dots, u_{m}\right\}$ in the $n$-dimensional vector space $V$:
\begin{enumerate}
  \item Find coordinate representation of $u_{1}, \dots, u_{m}$ with respect to some basis $S$ of $V$.

    This representation is an $n$-tuple or vectors in $\RR^{n}$.
  \item Find a basis for $\RR^{n}$ for the span of these $n$-tuples or vectors in $\RR^{n}$.

    This basis of $n$-tuples consists of the coordinate representation of the vectors n a basis for $span \left\{u_{1}, \dots, u_{m}\right\}$.
\end{enumerate}

\begin{lemma}
  Suppose $S = \left\{v_{1}, \dots, v_{n}\right\}$ is a basis for $V$, $\left\{u_{1}, \dots, u_{m}\right\} \subset v$, and $u \in V$.
  $u$ is a linear combination of $u_{1}, \dots, u_{m}$ iff the coordiante representation with respect to $S$ of $u$ (viewed as an $n$-tuple in $\RR^{n}$)
  is the same linear combination of the coordinate representations of $u_{1}, \dots, u_{m}$.
\end{lemma}

\begin{theorem}
  Suppose $S$ is a basis for the $n$-dimensional vector space $V$, and $u_{1}, \dots, u_{m}$ are vectors in $V$ such that the coord. representation of $u_{1}$
  is the row matrix $a_{i1}$ through $a_{in}$.a

  Let $A = \left[a_{ij}\right]$ so the rows of $A$ are the coord. representations of $u_{1}, \dots, u_{m}$.
  If $T$ is some set of vectors in $V$ whose coordinate representations with respects to $S$ form a basis for the row space of $A$ 
  as a subspace of $\RR^{n}$, then $T$ is a basis for $span \left\{u_{1}, \dots, u_{m}\right\}$.
\end{theorem}

\begin{remark}
  If $V$ is an $n$-dimensional vector space with basis $S$, then the coordinate representation with respect to $S$ gives a 
  one-to-one correspondence between vectors in $V$ and $n$-tuples in $\RR^{n}$.
\end{remark}

\begin{theorem}
  Let $S$ be a set of $k$ vectors in $\RR^{n}$, and let $A$ be the $k \times n$ matrix whose rows are the $n$-tuples in $S$.
  $S$ is linearly independent if and only if the row rank of $A$ is $k$, the number of elements in $S$.
\end{theorem}

\subsection{Column Space And Rank}
\begin{definition}
  The \vocab{column space} of a $p \times n$ matrix $A$ is the span of the columns of $A$ viewed as vectors in $\RR^{p}$.

  This is denoted $colspace(A)$.
\end{definition}

\begin{definition}
  The \vocab{column rank} of $A$ is the dimension of the column space of $A$.
\end{definition}

\begin{remark}
  If $A$ is $p \times n$, then the rows of $A$ are vectors in $\RR^{n}$ and the columns of $A$ are vectors in $\RR^{p}$.
\end{remark}

\begin{theorem}
  For any $p \times n$ matrix $A$, $rowrank(A) = colrank(A)$.
\end{theorem}

\begin{definition}[Rank]
  The \vocab{rank} of a matrix $A$ is the row rank of $A$, which is the same as the column rank.

  This is denoted $rank(A)$, or simply $r(A)$.
\end{definition}

\begin{theorem}
  The system $Ax = b$ is consistent iff $r(A) = r(\left[A \mid b\right])$.
\end{theorem}

\begin{lemma}
  For any vector $v$, $v_{1}, \dots, v_{n}$ in a vector space $V$, $v \in span \left\{v_{1}, \dots, v_{n}\right\}$
  iff $\dim(span \left\{v_{1}, \dots, v_{n}\right\}) = \dim(span \left\{v_{1}, \dots, v_{n}, v\right\})$.
\end{lemma}

\begin{remark}
  In practice, to determine if $r(A) = r(\left[A \mid b\right])$, we must row-reduced $\left[A \mid b\right]$.
\end{remark}

\begin{theorem}
  Suppose $Ax = b$ is a consistent system of $n$ variables. If $r(A) = k$, then the solution to the system are expressible in terms of $n - k$ arbitrary unknowns.
\end{theorem}

\begin{corollary}
  The homogeneous system $Ax = 0$ in $n$ variables has nontrivial solutions iff $r(A) \neq n$.
\end{corollary}

\subsection{Invertibility}
\begin{theorem}
  An $n \times n$ matrix $A$ is invertible iff $r(A) = n$. 
\end{theorem}

\begin{lemma}
  If $A$ and $B$ are $n \times n$ matrices satisfying $AB = I_{n}$, then the rows of $A$ are linearly independent, and $r(A) = n$.
\end{lemma}

\begin{lemma}
  If $A$ is an $n \times n$ matrix and $r(A) = n$, then there exists a matrix $C$ such that $CA = I_{n}$.
\end{lemma}

\begin{lemma}
  If $A$ and $B$ are $n \times n$ matrices such that $AB = I_{n}$, then $BA = I_{n}$.
\end{lemma}

\begin{corollary}
  A square matrix is invertible iff it can be transformed using elementary row operations to an upper triangular matrix with all diagonal element equal to $1$.
\end{corollary}





