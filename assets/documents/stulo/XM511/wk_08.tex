\section{Lectures 36-37}
\subsection{Diagonalization}
\begin{definition}
  A square matrix is \vocab{diagonalizable} if it is similar to a diagonal matrix.
\end{definition}

\begin{theorem}
  An $n \times n$ matrix $A$ is diagonalizable iff it has $n$ linearly independent eigenvectors.
\end{theorem}

\begin{theorem}
  If $A$ is similar to the diagonal matrix $D_{1}$ and $A$ is similar to the diagonal matrix $D_{2}$, then $D_{1}$ and $D_{2}$ have the same set of diagonal
  elements and each such element occurs with the same multiplicity in $D_{1}$ as it does in $D_{2}$.
\end{theorem}

\begin{theorem}
  Let $A$ be a square matrix. For each positive integer $k$, if $x_{1}, \dots, x_{k}$ are eigenvectors of $A$ with distinct eigenvalues $\lambda_{1}, \dots, \lambda_{k}$ 
  respectively, then $\left\{x_{1}, \dots, x_{k}\right\}$ is linearly independent.
\end{theorem}

\begin{corollary}
  If $A$ is $n \times n$ and $A$ has $n$ distinct real eigenvalues (given by $n$ distinct roots of the characteristic equation of $A$), then $A$ is diagonalizable.
\end{corollary}

\begin{theorem}
  Let $A$ be an $n \times n$ matrix whose real eigenvalues are $\lambda_{1}, \dots, \lambda_{k}$, and for each $j$,
  let $S_{\lambda_{j}}$ denote the eigenspace for the eigenvalue $\lambda_{j}$.

  $A$ is diagonalizable iff $\sum^{k}_{j = 1} \dim(S_{\lambda_{j}}) = 1$.
\end{theorem}

\begin{theorem}
  If $\lambda$ is an eigenvalue of the $n \times n$ matrix $A$, then 
  \begin{equation*}
    \dim(S_{\lambda}) = n - r(A - \lambda I_{n}).
  \end{equation*}
\end{theorem}
