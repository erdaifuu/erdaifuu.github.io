\section{Lectures 30-35}
\subsection{Determinants}
\begin{definition}
  The \vocab{determinant}, denoted $\det$, is a function from the set of square matrices to the set of scalars defined recursively by the following rules:
  \begin{itemize}
    \item If $A$ is the $1 \times 1$ matrix $A = ;[a_{11}]$, then $\det(A) = a_{11}$.
    \item If $A$ is an $n \times n$ matrix where $n > 1$, then
      \begin{equation*}
        \det(A) = \sum^{n}_{j = 1} (-1)^{1 + j} a_{1j} \det(\mu_{1j}(A)).
      \end{equation*}
  \end{itemize}
\end{definition}

\begin{definition}
  For a matrix $A = \left[a_{ij}\right]$, the scalar quantity $\det(\mu_{ij}(A))$ is called \vocab{a minor of} $A$, and 
  \begin{equation*}
    (-1)^{i + j} \det(\mu_{ij}(A)) 
  \end{equation*}
  is called the \vocab{cofactor} of $a_{ij}$.
\end{definition}

\begin{theorem}[Expansion by Cofactors]
  For any $n \times n$ matrix $A$, $\det(A)$ can be computed as follows:
  \begin{enumerate}[(1)]
    \item Choose any one row or any one column of $A$.
    \item Compute the cofactor of each element in the chosen row or column.
    \item Multiply each element in the chosen row or column by its cofactor, and sum the results.
  \end{enumerate}
\end{theorem}

\begin{corollary}
  If $A$ has a zero row or zero column, then $\det(A) = 0$.
\end{corollary}

\begin{theorem}
  If $A = \left[a_{ij}\right]$ is an upper triangular $n \times n$ matrix, then $\det(A)$ is the product of the diagonal elements
  \begin{equation*}
    \det(A) = a_{11}a_{22} \dots a_{nn}.
  \end{equation*}
\end{theorem}

A similar result holds for lower triangular matrices.

\begin{corollary}
  If $D$ is a diagonal matrix with entries $d_{1}, d_{2}, \dots, d_{n}$, then
  \begin{equation*}
    \det(D) = d_{1}d_{2} \dots d_{n}.
  \end{equation*}
\end{corollary}

\begin{theorem}[Geometric Properties of Determinants]
  The area of the parallelogram $P$ determined by the vectors $(a_{1}, a_{2})$ and $(b_{1}, b_{2})$ in $\RR^{2}$ is given by the formula
  \begin{equation*}
    Area(P) = \left|\det \left( \begin{bmatrix} a_{1} & b_{1} \\ a_{2} & b_{2}\end{bmatrix} \right)\right|.
  \end{equation*}
\end{theorem}

\begin{theorem}
  Let $C$ be the $2 \times 2$ matrix representation of the linear transformation $T : \RR^{2} \to \RR^{2}$ with respect to the standard basis, and let $P$ denote
  the parallelogram determined by the vectors $u$ and $v$ in $\RR^{2}$.

  If $T(P)$ denotes the parallelogram determined by $T(u)$ and $T(v)$, then
  \begin{equation*}
    Area(T(P)) = \left|\det(C)\right| \cdot Area(P).
  \end{equation*}
\end{theorem}

\begin{theorem}
  For any square matrix $A$, $\det(A) = \det(A^{T})$.
\end{theorem}

\begin{remark}
  Our goal for the next few lemmas is to eventually prove that ``determinants of a product is the product of the determinants''.
\end{remark}

\begin{lemma}
  If $A$ is $n \times n$ and $E$ is an $n \times n$ elementary matrix corresponding to an elementary row operation that interchanges two rows, then $\det(E) = -1$ and
  \begin{equation*}
    \det(EA) = -\det(A) = \det(E)\det(A).
  \end{equation*}
\end{lemma}

\begin{corollary}
  If the square matrix $A$ has two identical rows or two identical columns, then $\det(A) = 0$.
\end{corollary}

\begin{lemma}
  If $A$ is $n \times n$ and $E$ is the $n \times n$ elementary matrix corresponding to the elementary row operation that multiplies row $k$ by the nonzero scalar $\lambda$,
  then $\det(E) = \lambda$ and
  \begin{equation*}
    \det(EA) = \lambda \det(A) = \det(E) \det(A).
  \end{equation*}
\end{lemma}

\begin{corollary}
  For any scalar $\lambda$ and $n \times n$ matrix $A$,
  \begin{equation*}
    \det(\lambda A) = \lambda^{n} \det(A).
  \end{equation*}
\end{corollary}

\begin{lemma}
  If $A$ is $n \times n$ and $E$ is the elementary matrix corresponding to the elementary row operation that adds $\lambda$ times  row $s$  to row $r$, then $\det(E) = 1$ and
  \begin{equation*}
    \det(EA) = \det(A) = \det(E)\det(A).
  \end{equation*}
\end{lemma}

The three lemmas immediately imply
\begin{theorem}
  If $A$ is $n \times n$ and $E$ is an $n \times n$ elementary matrix, then
  \begin{equation*}
    \det(EA) = \det(E)\det(A).
  \end{equation*}
\end{theorem}

\begin{theorem}
  Let $A$ be a square matrix. $A$ is invertible iff $\det(A) \neq 0$.
\end{theorem}

\begin{theorem}
  For any $n \times n$ matrices $A$ and $B$, $\det(AB) = \det(A)\det(B)$.
\end{theorem}

\begin{corollary}
  If $A$ is an $n \times n$ invertible matrix, then $\det(A^{-1}) = (\det(A))^{-1}$.
\end{corollary}

\begin{corollary}
  Similar matrices have the same determinants.
\end{corollary}

\subsection{Eigenvectors and Eigenvalues}
\begin{definition}
  Let $T : V \to V$ be linear. If $u$ is nonzero and $\lambda$ is a scalar such that $T(u) = \lambda u$, then $u$ is called an \vocab{eigenvector} of $T$, and
  $\lambda$ is called an eigenvalue of $T$.
\end{definition}

\begin{theorem}
  Let $V$ be finite dimensional and $T : V \to V$ be linear. There exists a basis $B$ of $V$ such that the matrix representation of $T$ with respect to $V$ is a diagonal
  matrix iff there exists a basis of $V$ containing of only eigenvectors of $T$.
\end{theorem}

\begin{definition}[Eigenvectors and Eigenvalues from Matrices]
  Let $A$ be an $n \times n$ matrix. If $x$ is a nonzero $n$-tuple in $\RR^{n}$ viewed as a column matrix and $\lambda$ is a scalar such that $Ax = \lambda x$,
  then $x$ is called an \vocab{eigenvector} of $A$, and $\lambda$ is called an \vocab{eigenvalue} of $A$.
\end{definition}

\begin{theorem}
  $u$ is an eigenvector with eigenvalue $\lambda$ for the linear transformation $T : V \to V$ iff for any basis $C$ of $V$ and matrix representation $A^{C}_{C}$ of $T$,
  $(u)_{C}$ is an eigenvector $\lambda$ for the matrix $A^{C}_{C}$.
\end{theorem}

\begin{proof}
  \begin{align*}
    T(u) = \lambda u &\iff (T(u))_{C} = (\lambda u)_{C} \\ 
                     &\iff (T(u))_{C} = \lambda \cdot (u)_{C} \\ 
                     &\iff A^{C}_{C} (u)_{C} = \lambda \cdot (u)_{C}.
  \end{align*}
\end{proof}

\begin{theorem}
  For a linear transformation $T : V \to V$ with eigenvalue $\lambda$, the set 
  \begin{equation*}
    S_{\lambda} = \left\{v \in V \mid T(v) = \lambda v\right\}
  \end{equation*}
  which equals $\left\{0\right\} \cup \left\{v \in V \mid v \text{ is an eigenvector with eigenvalue } \lambda\right\}$.
\end{theorem}

\begin{definition}
  For a transformation $T : V \to V$ with eigenvalue $\lambda$,
  \begin{equation*}
    S_{\lambda} = \left\{v \in V \mid T(v) = \lambda v\right\}
  \end{equation*}
  which equals $\left\{0\right\} \cup \left\{v \in V \mid v \text{ is an eigenvector with eigenvalue } \lambda\right\}$, is called the \vocab{eigenspace} of $T$ for the eigenvalue $\lambda$.
\end{definition}

\subsection{Finding Eigenvectors and Eigenvalues}
\begin{theorem}
  $\lambda$ is an eigenvalue of the $n \times n$ matrix $A$ iff $\det(A - \lambda I_{n}) = 0$.
\end{theorem}

\begin{definition}
  The equation $\det(A - \lambda I_{n}) = 0$ is called the \vocab{characteristic equation} of the matrix $A$.
\end{definition}

If 
\begin{equation*}
  A = \begin{bmatrix}
    a_{11} & a_{12} & \dots & a_{1n} \\
    a_{21} & a_{22} & \dots & a_{2n} \\
    \vdots & \vdots & \ddots & \vdots \\
    a_{n1} & a_{n2} & \dots & a_{nn}
  \end{bmatrix},
\end{equation*}
then 
\begin{equation*}
    A - \lambda I_{n}  =
\begin{bmatrix}
    a_{11} - \lambda & a_{12} & \dots & a_{1n} \\
    a_{21} & a_{22} - \lambda & \dots & a_{2n} \\
    \vdots & \vdots & \ddots & \vdots \\
    a_{n1} & a_{n2} & \dots & a_{nn} - \lambda
  \end{bmatrix},
\end{equation*}

\begin{example}
  Let $A = \begin{bmatrix}
    1 & 2 \\ -4 & 7
  \end{bmatrix}$. Find all eigenvalues of $A$ and each corresponding eigenspace.


  To do this, we need to solve the characteristic equation $\det(A - \lambda I_{n}) = 0$.
  \begin{equation*}
    A - \lambda I_{2} = \begin{bmatrix}
      1 - \lambda & 2 \\ -4 & 7 - \lambda.
    \end{bmatrix}
  \end{equation*}
  Solving this, we get the determinant as $(\lambda - 3) ( \lambda - 5)$, and so this has solutions $\lambda = 3$ and $\lambda = 5$, implying they are the eigenvalues of $A$.

  To find the eigenspace of $3$, we must find all $x$ such that $Ax = 3x$. So we solve the matrix equation $(A - 3 I_{2})x = 0$.
  We see that this equation is solved iff $x_{1} = x_{2}$, so $\begin{bmatrix}
    1 \\ 1
  \end{bmatrix}$ is one eigenvector with eigenvalue $3$, and it is the basis for the eigenspace $S_{3}$.
\end{example}

\subsection{Properties of Eigenvectors and Eigenvalues}
\begin{theorem}
  If $A$ and $B$ are similar $n \times n$ matrices, then they have the same characteristic equation and therefore the same eigenvalues.
\end{theorem}

\begin{theorem}
  For an upper or lower triangular matrix, the eigenvalues are the diagonal elements.
\end{theorem}

\begin{theorem}
  An $n \times n$ matrix $A$ is singular iff it has a zero eigenvalue.
\end{theorem}

For the theorem ``The following statements are equivalent for an $n \times n$ matrix $A$'', we can now add ``every eigenvalue of $A$ is nonzero'' to the list.
For the theorem ``
\begin{theorem}
  Let $A$ be invertible. If $x$ is an eigenvector of $A$ with eigenvalue $\lambda$, then $x$ is an eigenvector of $A^{-1}$ with eigenvalue $\frac{1}{\lambda}$.
\end{theorem}

\begin{theorem}
  If $x$ is an eigenvector of $A$ with eigenvalue $\lambda$, then
  \begin{enumerate}
    \item For each scalar $k$, $x$ is an eigenvector of $kA$ with eigenvalue $k \lambda$.
    \item For each positive integer $n$, $x$ is an eigenvector of $A^{n}$ with eigenvalue $\lambda^{n}$.
  \end{enumerate}
\end{theorem}

\begin{definition}
  If $A = \left[a_{ij}\right]$ is an $n \times n$ matrix, then the \vocab{trace} of $A$, denoted $tr(A)$, is the scalar defined by
  \begin{equation*}
    tr(A) = \sum^{n}_{i = 1} a_{ii}.
  \end{equation*}
  So the trace of $A$ is the sum of its diagonal elements.
\end{definition}

\begin{theorem}
  Suppose $A$ is an $n \times n$ matrix with eigenvalues $\lambda_{1}, \dots, \lambda_{n}$ listed with multiplicity,
  \begin{enumerate}
    \item $tr(A) = \lambda_{1} + \lambda_{2} + \dots + \lambda_{n}$.
    \item $\det(A) = \lambda_{1} \lambda_{2} \dots \lambda_{n}$.
  \end{enumerate}
\end{theorem}



